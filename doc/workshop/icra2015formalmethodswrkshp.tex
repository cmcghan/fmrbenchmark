\documentclass{amsart}
\usepackage{hyperref}
\usepackage{graphicx}

\DeclareMathOperator{\Galways}{\mathbf{G}}
\DeclareMathOperator{\Feventually}{\mathbf{F}}
\DeclareMathOperator{\Xnext}{\mathbf{X}}
\DeclareMathOperator{\Uuntil}{\mathbf{U}}

\theoremstyle{definition}
\newtheorem{example}{Example}[section]


\begin{document}
\title{Workshop on Formal Methods For Robotics: Benchmarks and Evaluation Metrics}
\author{Scott C. Livingston}
\author{Vasumathi Raman}

\begin{abstract} %200 words on the money
The term ``formal methods'' refers broadly to techniques for the verification and automatic synthesis of systems that satisfy desirable properties exactly or within some statistical tolerance.  Historically developed for concurrent software, recent work has brought these methods to bear on motion planning in robotics. Challenges specific to robotics, such as uncertainty and
real-time constraints, have motivated extensions and entirely novel treatments which draw on ideas from automata theory, logic, model checking, hybrid systems and control.  However, compared to other areas of robotics research, demonstrations of formal methods have been surprisingly small-scale. In addition, there is a lack of benchmarks and metrics for evaluating novel algorithms.

This half-day workshop will bring together leading researchers in robotics, formal methods and hybrid systems, including those from industry. The discussions will be focused on metrics for evaluating the state of the art of formal methods for synthesis and verification of robotic systems, including the construction of appropriate benchmarks. This workshop will build on the success of its predecessors at ICRA 2009, 2010, CAV 2011, RSS 2013 and RSS 2014. Attendees will also be exposed to insights gained from the First Challenge on Formal Methods for Robotics, which has been conditionally accepted to ICRA 2015.

\end{abstract}
\maketitle


\section{Tentative Schedule}
We anticipate the following decomposition of a 3.5-hour workshop (9am-12:30pm):
\begin{itemize}
\item 2 x 20 minutes review the state of the art in formal methods by invited speakers 
\item 1 x 20 minute presentation from Formal Methods for Robotics Challenge organizers on the use of benchmarks and evaluation metrics for the challenge 
\item 10 minute break
\item 6 x 10 minute presentations from Formal Methods for Robotics Challenge participants
\item 10 minute break
\item 60 minutes of open discussion
\end{itemize}


[TODO: List of invited speakers]


\section{Participation}
We will solicit participation through common venues, such as the robotics-worldwide mailing list. In addition, we will contact directly research groups whose work is relevant to this workshop. Moreover, we expect most participating teams in the Formal Methods for Robotics Challenge to attend and contribute to this workshop. 

In addition to academic participation, we anticipate continued participation from industry, including Rethink Robotics and Toyota Research, who participated in the workshop at RSS 2014. 

[TODO: Estimate the number of participants that you expect, and provide a 
basis for this estimate, such as attendance at similar events in past years. If you have organized similar 
events in the past, list them and provide the estimated attendance at each. -- 500 WORD MAXIMUM]

\section{Plan for interaction among participants}
We expect that the FMR Challenge will provide ample fodder for discussion, both among challenge participants and outsiders. We see as a concrete goal of this workshop a collection of opinions on the successes and failures of the various benchmark problems proposed the the challenge, and suggestions for improving them. We also hope to incite participants to leave the workshop determined to construct more suitable benchmarks and evaluation metrics based on these discussions.

Discussion
[TODO: Describe how you will promote active discussion. Steps that result in participants staying for the entire
event and that increase the level of interaction between established experts and early-career researchers 
are particularly welcome. -- 500 WORD MAXIMUM]

\section{Equipment}
We do not foresee any requirements other than a projector setup.


\bibliographystyle{abbrv}
\bibliography{fmrchallenge.bib}

\end{document}
