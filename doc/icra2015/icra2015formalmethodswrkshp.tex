\documentclass{amsart}
\usepackage{hyperref}
\usepackage{graphicx}


\begin{document}
\title{Workshop on Formal Methods For Robotics: Benchmarks and Evaluation Metrics}
\author[Scott C. Livingston]{Scott C. Livingston$^{1}$}
\author[Vasumathi Raman]{Vasumathi Raman$^{2}$}
\thanks{$^{1}$\textbf{primary contact:} S. C. Livingston is with the Control and Dynamical Systems program, California Institute of Technology, Pasadena CA 91125, USA (phone: +1 626 395 3531, email: \href{mailto:slivingston@cds.caltech.edu}{slivingston@cds.caltech.edu}, url: \url{http://scottman.net}).}%
\thanks{$^{2}$V. Raman is with the Department of Computing and Mathematical Sciences, California Institute of Technology, Pasadena CA 91125, USA (phone: +1 607 216 8623, email: \href{mailto:vasu@caltech.edu}{vasu@caltech.edu}, url: \url{http://www.cms.caltech.edu/~vasu/}).}%

\begin{abstract} %200 words on the money
The term ``formal methods'' refers broadly to techniques for the verification and automatic synthesis of systems that satisfy desirable properties exactly or within some statistical tolerance.  Historically developed for concurrent software, recent work has brought these methods to bear on motion planning in robotics. Challenges specific to robotics, such as uncertainty and
real-time constraints, have motivated extensions and entirely novel treatments which draw on ideas from automata theory, logic, model checking, hybrid systems and control.  However, compared to other areas of robotics research, demonstrations of formal methods have been surprisingly small-scale. In addition, there is a lack of benchmarks and metrics for evaluating novel algorithms.

This half-day workshop will bring together leading researchers in robotics, formal methods and hybrid systems, including those from industry. The discussions will be focused on metrics for evaluating the state of the art of formal methods for synthesis and verification of robotic systems, including the construction of appropriate benchmarks. This workshop will build on the success of its predecessors at ICRA 2009, 2010, CAV 2011, RSS 2013 and RSS 2014. Attendees will also be exposed to insights gained from the First Challenge on Formal Methods for Robotics, which has been conditionally accepted to ICRA 2015.

\end{abstract}
\maketitle


\section{Type and duration}

We propose a half-day workshop.

Because some of the discussion will concern results of the First Challenge on
Formal Methods for Robotics (\url{http://www.fmrchallenge.org}), which is
scheduled for 26--28 May at ICRA, it is best that the workshop occur on the
later day of workshops, 30 May 2015.

\section{URL}

We plan to host a page about the workshop at \url{http://www.fmrchallenge.org/icra2015workshop}.   It is not active at the time of proposal submission.


\section{Content}

This workshop follows a series of workshops focusing on formal methods and
robotics that have been held at various conferences in recent years:
\href{http://verifiablerobotics.com/ICRA09/index.html}{ICRA 2009},
\href{http://verifiablerobotics.com/ICRA10/index.html}{2010},
\href{http://verifiablerobotics.com/CAV11/index.html}{CAV 2011} (Computer-Aided
Verification), and \href{http://verifiablerobotics.com/RSS13/index.html}{RSS
  2013}, \href{http://verifiablerobotics.com/RSS14/index.html}{2014} (Robotics:
Science and Systems).  These previous events have tended toward broad inclusion
of relevant work, motivated in part by the nascency of the field.  We propose to
focus this manifestation at ICRA~2015 on benchmarks and metrics for evaluation.
Our motivation is twofold.  First, a relevant robotics challenge at ICRA has
been conditionally accepted, and a venue for discussing preparations and results
will be of broad interest since the challenge will be the first of its kind.
Second, a common criticism of formal methods for robotics is unremarkable
demonstrations and unclear potential for large, practical applications, and a
workshop focused on benchmarks and evaluation metrics will directly address this
by creating a venue for discussion by active researchers in the field.

\subsection{Tentative schedule}
We anticipate the following decomposition of a 3.5-hour workshop (e.g., during 09:00--12:30):
\begin{itemize}
\item 3 $\times$ 20 minutes review the state of the art, describe open problems, and suggest future directions of formal methods for robotics by invited speakers (listed below)
\item 1 $\times$ 20 minute presentation from Formal Methods for Robotics Challenge organizers on the use of benchmarks and evaluation metrics for the challenge 
\item 10 minute break
\item 6 $\times$ 10 minute presentations from Formal Methods for Robotics Challenge participants
\item 10 minute break
\item 50 minutes of open discussion
\end{itemize}
We are also considering shortening the ``open discussion'' and preceding it
with a question-and-answer panel of the invited speakers.

\subsection{Invited speakers}
\begin{itemize}
\item Julia Badger (Braman) (NASA Johnson Space Center)
\item Lydia E. Kavraki (Rice U.)
\item George J. Pappas (U.~Penn)
\end{itemize}
All of the above are still pending.  In case those already invited cannot do it,
additional people to invite include the following:
\begin{itemize}
\item Jonathan P. How (MIT)
\item Paulo Tabuada (UCLA)
\item Sertac Karaman (MIT)
\end{itemize}

\section{Plan to solicit participation}
We will solicit participation through common venues, such as the
robotics-worldwide mailing list, as well as by directly contacting research
groups whose work is relevant to this workshop. Moreover, we expect most
participating teams in the Formal Methods for Robotics Challenge to attend and
contribute to this workshop.

Since this workshop will be framed in part as a continuation of a series, we
expect the workshop title to be easily recognized by many potential
participants, whether or not they are involved in the challenge.  Private
communication with Kress-Gazit, one of the organizers of previous ``formal
methods for robotics'' workshops, estimates there were about 50 participants at
RSS~2014 and 30 at RSS~2013.  We will seek lists of email addresses of attendees
from these as another medium for solicitation.

In addition to academic participation, we anticipate continued participation from industry, including Rethink Robotics and Toyota Research, who participated in the workshop at RSS 2014. 


\section{Plan to encourage interaction among participants}
We expect that the First Challenge on Formal Methods for Robotics will provide ample fodder for discussion.  To ensure accessibility to a broad ICRA audience, the presentation by the challenge organizers will provide an overview about it.  We see as a concrete goal of this workshop a collection of opinions on the successes and failures of the various benchmark problems proposed in the challenge, and suggestions for improving them. We also hope to incite participants to leave the workshop determined to construct more suitable benchmarks and evaluation metrics based on these discussions.



\section{Equipment}
We do not foresee any requirements other than a projector setup.
%% chalkboard

\section{Support of an IEEE RAS Technical Committee}

(none)


\end{document}
